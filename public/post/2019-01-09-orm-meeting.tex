\documentclass[]{article}
\usepackage{lmodern}
\usepackage{amssymb,amsmath}
\usepackage{ifxetex,ifluatex}
\usepackage{fixltx2e} % provides \textsubscript
\ifnum 0\ifxetex 1\fi\ifluatex 1\fi=0 % if pdftex
  \usepackage[T1]{fontenc}
  \usepackage[utf8]{inputenc}
\else % if luatex or xelatex
  \ifxetex
    \usepackage{mathspec}
  \else
    \usepackage{fontspec}
  \fi
  \defaultfontfeatures{Ligatures=TeX,Scale=MatchLowercase}
\fi
% use upquote if available, for straight quotes in verbatim environments
\IfFileExists{upquote.sty}{\usepackage{upquote}}{}
% use microtype if available
\IfFileExists{microtype.sty}{%
\usepackage{microtype}
\UseMicrotypeSet[protrusion]{basicmath} % disable protrusion for tt fonts
}{}
\usepackage[margin=1in]{geometry}
\usepackage{hyperref}
\hypersetup{unicode=true,
            pdftitle={ORM Paper - first meeting of the year},
            pdfauthor={Tim Ballard},
            pdfborder={0 0 0},
            breaklinks=true}
\urlstyle{same}  % don't use monospace font for urls
\usepackage{graphicx,grffile}
\makeatletter
\def\maxwidth{\ifdim\Gin@nat@width>\linewidth\linewidth\else\Gin@nat@width\fi}
\def\maxheight{\ifdim\Gin@nat@height>\textheight\textheight\else\Gin@nat@height\fi}
\makeatother
% Scale images if necessary, so that they will not overflow the page
% margins by default, and it is still possible to overwrite the defaults
% using explicit options in \includegraphics[width, height, ...]{}
\setkeys{Gin}{width=\maxwidth,height=\maxheight,keepaspectratio}
\IfFileExists{parskip.sty}{%
\usepackage{parskip}
}{% else
\setlength{\parindent}{0pt}
\setlength{\parskip}{6pt plus 2pt minus 1pt}
}
\setlength{\emergencystretch}{3em}  % prevent overfull lines
\providecommand{\tightlist}{%
  \setlength{\itemsep}{0pt}\setlength{\parskip}{0pt}}
\setcounter{secnumdepth}{0}
% Redefines (sub)paragraphs to behave more like sections
\ifx\paragraph\undefined\else
\let\oldparagraph\paragraph
\renewcommand{\paragraph}[1]{\oldparagraph{#1}\mbox{}}
\fi
\ifx\subparagraph\undefined\else
\let\oldsubparagraph\subparagraph
\renewcommand{\subparagraph}[1]{\oldsubparagraph{#1}\mbox{}}
\fi

%%% Use protect on footnotes to avoid problems with footnotes in titles
\let\rmarkdownfootnote\footnote%
\def\footnote{\protect\rmarkdownfootnote}

%%% Change title format to be more compact
\usepackage{titling}

% Create subtitle command for use in maketitle
\newcommand{\subtitle}[1]{
  \posttitle{
    \begin{center}\large#1\end{center}
    }
}

\setlength{\droptitle}{-2em}

  \title{ORM Paper - first meeting of the year}
    \pretitle{\vspace{\droptitle}\centering\huge}
  \posttitle{\par}
    \author{Tim Ballard}
    \preauthor{\centering\large\emph}
  \postauthor{\par}
      \predate{\centering\large\emph}
  \postdate{\par}
    \date{2019-01-09 T21:13:14-05:00}


\begin{document}
\maketitle

Attendees: Andrew, Hector

\hypertarget{items-for-discussion}{%
\section{Items for discussion}\label{items-for-discussion}}

\hypertarget{finalising-the-model}{%
\subsubsection{Finalising the model}\label{finalising-the-model}}

\begin{itemize}
\item
  Choose final model (need to pick which performance function works
  best).

  \begin{itemize}
  \tightlist
  \item
    Model 1:
  \end{itemize}
\end{itemize}

\[skill = 1 - e^{-\delta \times practice}\]

\[\delta_{performance} = \frac{k_1 + k_2 \times skill}{1 + e^{-(k_3 + k_4 \times effort + k_5 \times skill)}}\]

\begin{itemize}
\item
  \begin{itemize}
  \item
    Issues:

    \begin{itemize}
    \item
      Model fits well, but looks too much like a regression equation.
    \item
      Skill comes into the equation in multiple ways (i.e., it
      influences the intercept and slope). This may be required to
      reproduce the K \& A figure, but is not parsimonious and not
      necessary to fit the data.
    \end{itemize}
  \item
    Model 2:
  \end{itemize}
\end{itemize}

\[skill = skill_{assymptote} - (skill_{assymptote} - skill_0)^{-\delta \times practice}\]

\[\delta_{performance} = \frac{skill}{1 + e^{-(k_1 + k_2 \times effort)}}\]

\begin{itemize}
\item
  \begin{itemize}
  \item
    Issues:
  \item
    This model does not influence the slope of the function, but that's
    probably okay.
  \item
    The CIs on the skill trajectory are very wide.
  \end{itemize}
\item
  \textbf{Action: Tim to rerun with more constrained priors. Check to
  see if the credible interval on skill has narrowed.}
\item
  Which parameters to manipulate in simulation study?

  \begin{itemize}
  \tightlist
  \item
    \emph{not discussed}
  \end{itemize}
\item
  Lower case greek letters for each parameter or should effort baseline,
  initial effort, initial goal be the word with some subscript?

  \begin{itemize}
  \tightlist
  \item
    \emph{not discussed}
  \end{itemize}
\item
  Need to compare the models to simpler alternatives to establish
  necessity. Could we compare the models to alternatives where either
  effort or goal do not update at all? Are these two simple? What other
  ones.

  \begin{itemize}
  \tightlist
  \item
    \textbf{Action: Tim to try to implement linear models of the
    observed variables as alternative models. Demonstrates that the
    mechanism is more complex than a simple linear model, also
    demonstrates that people can use this method to implement linear
    models as well.}
  \end{itemize}
\end{itemize}

\hypertarget{finishing-the-paper}{%
\subsubsection{Finishing the paper}\label{finishing-the-paper}}

\begin{itemize}
\item
  Where does the hierarchical structure fit in to the modeling? Does it
  go in Step 2 (Model fitting) or is it a 4th step that involves
  interpretting the model parameters. On the one hand, the hierarchical
  structure always needs to be a consideration for model fitting, but
  this might disrupt the flow of the paper.

  \begin{itemize}
  \tightlist
  \item
    \textbf{Action: Include section on model structure and parameter
    interpretation after model comparisons. The idea is that one might
    want to interpret the model parameters and to do this the researcher
    must consider how these parameters are implemented in the model.
    Figures from original version with distributions and individual
    differences.}
  \end{itemize}
\item
  Do we need a recent grad student to look over paper?

  \begin{itemize}
  \tightlist
  \item
    \emph{not discussed}
  \end{itemize}
\item
  Hector to look over code and help with documentation?

  \begin{itemize}
  \tightlist
  \item
    \textbf{Action: Tim to prepare code for Hector over next few working
    days and let Hector know when it is ready. Hector to go through and
    make sure documentation is accessible.}
  \end{itemize}
\item
  How do we bring Mark back in?

  \begin{itemize}
  \tightlist
  \item
    \textbf{Action: Tim to email Mark (and Andrew and Hector) asking if
    Mark can look over first three sections. Andrew to look over
    document after Mark.}
  \end{itemize}
\end{itemize}


\end{document}
